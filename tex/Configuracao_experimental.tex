%The tests were performed using the \gls{BUTPPG}~\cite{butppg}. In such a database, it was utilized a smartphone to record 48 \acrshort{ppg} signals of 12 \acrlong{sbjs} (\acrshort{sbjs}) index fingers, in such a manner that 3 records were extracted when the subject was sat down and a single record was extracted when he or she was walking \cite{butppg}. This database can be consumed through Physionet~\cite{physionet} interface via \texttt{wfdb} package. The proposed method was implemented in Python. Implementations of \acrshort{gaf}, \acrshort{mtf} and \acrshort{rp} were provided by PyTS \cite{pyts} library. The PyTorch library~\cite{pytorch} was used to perform the training and the classification operations. The models used in this study are listed in Table~\ref{tab:results} and provided by TorchVision. Hyper-parameters optimization was performed using the Optuna library~\cite{optuna}. For each model, 50 Optuna trials were performed for fitting and validating the \acrshort{ml} model with median pruner to avoid excessive computation of trial epochs that do not show hope of better results.

%After the optimization, the metrics of the best trial were evaluated in the testing dataset to assess the selected hyperparameters' process quality. With selected hyper-parameters for each \acrshort{ml} model in hands, the dataset was divided into folds using the cross-validation \acrfull{loso} re-sampling method, into pieces matching each of the 12 \acrshort{sbjs}. For each fold, the smaller split was used as the testing dataset for the evaluation of the model's metrics, while the bigger split was subdivided into the training dataset, of size 7 \acrshort{sbjs}; and into the validation dataset, of size 4 \acrshort{sbjs}. Applying such an experimental setup allowed the generation of results concerning the metrics present in Table~\ref{tab:results}, where, for each projection method, all models were seen as samples of a statistics population possessing 5 values corresponding to the mean of all 12 folds of each metric.

% The hyper-parameters selection can be a computationally expensive process, considering large sets of possible parameters to be searched exhaustively. Hence, Optuna, a hyper-parameter optimization framework based on heuristic search with pruning \cite{optuna}, was used. But, before selecting such parameters, it was needed to separate the dataset in train, with 7 \acrshort{sbjs}; validation, with 2 \acrshort{sbjs}; and test, with 3 \acrshort{sbjs}; splits.  Later, it was optimized, in 50 Optuna trials, using the train dataset for fitting the \acrshort{ml} model and the validation dataset for evaluating the objective function value. Furthermore, it was used Optuna's median pruner to avoid excessive computation of trial epochs that do not show hope of better results, if compared to previous trials. Such an optimization process allowed to find the set of best hyper-parameters for each model, which, in this case, contain only the learning rate. 

%In order to evaluate those projection methods, the following metrics were considered accuracy score, the proportion of hits $Accuracy = \frac{(TP+TN)}{TP+TN+FP+FN}$; precision score, the proportion of correct positive guesses, $Precision = \frac{TP}{TP+FP}$; recall score, the proportion of found positive samples, $Recall = \frac{TP}{TP+FN}$; and F1 score, the harmonic mean between the precision score and the recall score, $F1 = 2 \cdot \frac{Precision \cdot Recall}{Precision + Recall}$. Additionally, it was measured the Cohen Kappa score, the degree of agreement of annotators for a classification problem, $CohenKappa = 1-\frac{1-p_o}{1-p_e}$, where $p_o$ is the proportion of observed concordance and $p_e$ is the probability of concordance between all annotators. In the supervised binary classification domain, only two annotators, corresponding to the predicted and the real labels, and two classes, corresponding to positive or negative, are considered, in a manner that the confusion matrix can be used directly to evaluate the score, by the formula $CohenKappa=\frac{2\cdot (TP\cdot TN - FP \cdot FN)}{(TP+FP)\cdot(FP+TN)+(TP+FN)\cdot(FN+TN)}$.

% Those metrics were evaluated for 70 \acrshort{ml} models implemented in \href{https://pytorch.org/}{PyTorch}, the ones listed on Table~ \ref{tab:results}
% For such networks, the same experiment framework was applied, highlighting three major process blocks: dataset construction, when the database was loaded and transformed; hyperparameters selection, when, for each \acrshort{ml} model, a search was done to try to find the set of its best hyperparameters for the built dataset; and projection metrics evaluation, when each model, in conjunction with its set of best hyper-parameters found, was tested, generating the metrics present in this article. %(the framework is shown in figure \ref{figure:framework})
% Also, hyper-parameters selection and projections evaluation apply training and testing cycles.  To clarify those processes, they will be described in detail in the following sections.


%\begin{figure*}[t]
%    \centering
%    \includegraphics[width=\linewidth]{imgs/framework2.pdf}
%    \caption{Flowchart representing the experimental framework. Gray elements are entities, while blue elements are processes.}
%    \label{figure:framework}
%\end{figure*}

% \subsection{Training and Testing}

% Machine learning tasks involve 2 main steps: training and testing. Such a format was applied several times on the hyper-parameters selection process, for testing certain sets of hyper-parameters; and the projections evaluation process, for evaluating the model and its set of hyper-parameters efficiency. That core task was done using \href{https://pytorch.org/}{PyTorch} python library, which automatically computes the gradients based on the user's implementation \cite{pytorch}. Also, it allows parallel GPU processing to be done, by applying tensor-based operations to batches of the dataset, which size used in this experiment was the whole dataset. 

% The training process parameters also involve an optimizer and and loss function. The optimizer algorithm to be used was Adam\cite{adam}, while the Cross-Entropy Loss function was used, %\cite{cross-entropy-loss} 
% both implemented in \href{https://pytorch.org/}{PyTorch}. Moreover, the training process needs to stop at some point, fact that was done using an early stopping approach, computing the median absolute deviation, deviation metric robust to outliers, %\cite{?}
% of a window of the last 10 loss function values on the validation dataset, stopping the training process when such windows converge to, at least, a deviation of value 1. When the training process stopped, the best result found was chosen.

% \subsection{Dataset construction}

% In order to provide data to train and test \acrshort{ml} models, the Brno University of Technology Smartphone PPG Database was used as a starting point. In such a database, it was utilized a smartphone to record 48 \acrshort{ppg} signals of 12 \acrlong{sbjs} (\acrshort{sbjs}) index fingers, in such a manner that 3 records were extracted when the subject was sat down and a single record was extracted when he or she was walking \cite{butppg}. Additionally, other refinement procedures were done to perfect that database, such as cropping 20 seconds of the recording, leaving the middle 10 seconds intact\cite{butppg}. And, finally, the database is publicly available on Physionet through the link \url{https://physionet.org/content/butppg/1.0.0/}. 

% However, the database is in \acrfull{wfdb} format \cite{butppg}, one dimensional signal format in function of time, %\cite{wfdb}
% letting yet to be done the application of projection methods. For such purpose, PyTS was used, a Python package developed for time series classification, containing the desired projection algorithms, \acrshort{gaf}, \acrshort{mtf} and \acrshort{rp} \cite{pyts}, with its implementations of PyTS version 0.13.0 stored in \href{https://zenodo.org/}{Zenodo} repository \cite{pyts-v0.13.0}.
    
% With such a package, it was possible to produce a dataset containing all projections as 2D images with 3 channels with shape $3 \times siglen^2 \times siglen^2$, where $siglen$ is the signal length. For each projection, its matrix was repeated on every channel of the image, while the \gls{PM} filled every channel with each of the 3 mentioned projections. Finally, it was necessary to resize every image for each \acrshort{ml} model input shape, for such purpose that it was used \href{https://pytorch.org/}{PyTorch} resize transform with bi-linear interpolation. After those procedures, the dataset was ready for the following processes, as the hyper-parameters selection.

% \subsection{Hyper-parameters selection}

% The hyper-parameters selection can be a computationally expensive process, considering large sets of possible parameters to be searched exhaustively. Hence, Optuna, a hyper-parameter optimization framework based on heuristic search with pruning \cite{optuna}, was used. But, before selecting such parameters, it was needed to separate the dataset in train, with 7 \acrshort{sbjs}; validation, with 2 \acrshort{sbjs}; and test, with 3 \acrshort{sbjs}; splits.  Later, it was optimized, in 50 Optuna trials, using the train dataset for fitting the \acrshort{ml} model and the validation dataset for evaluating the objective function value. Furthermore, it was used Optuna's median pruner to avoid excessive computation of trial epochs that do not show hope of better results, if compared to previous trials. Such an optimization process allowed to find the set of best hyper-parameters for each model, which, in this case, contain only the learning rate. 

%\subsection{Projection Metrics Evaluation}

%After the optimization, the metrics of the best trial were evaluated in the test dataset to assess the selected hyperparameters' process quality. With selected hyper-parameters for each \acrshort{ml} model in hands, the dataset was divided into folds using the cross-validation \acrfull{loso} re-sampling method, into pieces matching each of the 12 \acrshort{sbjs}. For each fold, the left split was used as the test dataset for the evaluation of the model's metrics, while the other split was subdivided into the train dataset, of size 7 \acrshort{sbjs}; and into the valid dataset, of size 4 \acrshort{sbjs}. Applying such an experimental setup allowed to generate results with respect to the before-mentioned metrics, where, for each projection method, all models are seen as samples of a statistics population possessing 5 values corresponding to the metrics.

