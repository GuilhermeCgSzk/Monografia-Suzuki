%Signal quality assessment is a fundamental component in the area of health monitoring applications, especially in the context of health applications in wearable devices%~\cite{lucafo2022}. These wearable devices serve as instruments for the continuous measurement of vital signs and health parameters, yet their efficacy is contingent upon the rigorous evaluation of the signals they capture. Although those devices are usually designed for long periods of monitoring, they use an optical method to obtain a \acrfull{ppg}, signal type that may deteriorate due to many factors, including motion artifacts or changes in illumination, thereby requiring signal quality assessment mechanisms to check its reliability. The quality of the data derived from these devices plays a decisive function in the efficacy of machine learning and artificial intelligence algorithms, frequently deployed for health data analysis and decision-making. Thus, by ensuring signal quality, wearable health applications can mitigate false alarms, deliver tailored healthcare recommendations, bolster research endeavors, and enhance device adoption and long-term utilization, consequently advancing the landscape of healthcare and patient well-being. In this work, we propose a solution for the problem of assessing the quality of 1D %\acrshort{ppg} signals by projecting them into 2D images, to take advantage of %\gls{CV} techniques, which requires bi-dimensional data.
