%This work presented a study on \gls{SQA} for \gls{ppg} signals. The investigated approach consists of projecting \gls{1D} signals onto \gls{2D} images using \acrshort{rp}, \acrshort{gaf}, and \acrshort{mtf} methods. Further to these methods, we also proposed the use of a mixture of them. Results indicate that \acrshort{rp} and \gls{PM} projection methods achieved better performance than \acrshort{gaf} and \acrshort{mtf} methods, while \acrshort{rp} and \gls{PM} had similar results. While the set of machine learning models was large, the number of data available in the \gls{BUTPPG} dataset was small and unbalanced, not giving very definitive results. Therefore, the experiment needs to be replicated on a large dataset, either by using data augmentation techniques to balance and enlarge the BUT PPG dataset or by utilizing another dataset with more samples. 

% Conclusão a respeito do método proposto
This thesis lead to a series of relevant conclusions. The experiments of the Chapter \ref{Experimentos} gives insights about the importance of the novel proposed projection method. One important result is that the \gls{PMix} method was more prevalent in the best results than the other projection methods, since, among the projection-based ensembles in the best results Table \ref{tab:Averages_of_Best Models}, all occurrences except one used the \gls{PMix} method. That leads to the conclusion that we can use the proposed method to improve the performance of a set of available isolated projection methods. However, it should be considered that that method increase the primary memory requirements to the summation of the individual projection sizes.	  

% Conclusão a respeito dos métodos de projeção em geral
The same experiments also allow us to notice the competence of the projection methods in general. An important experiment result on that regard is that, in the best models Table \ref{tab:Averages_of_Best Models}, the \gls{PMix} method with the Wide ResNet surpassed the baseline time series classification models for the Cohen Kappa score, indicating that a projection-based approach was capable of being very accurate for both binary \gls{SQI} classes. That highlights the viability of projection-based approach in par with other time series classifiers, but the convetional time series classifiers did achieved a better F1 and Precision scores, hinting that they worked better for the positive \gls{SQI} class. Additionally, the projection-based plus \gls{CV} method is computationally expensive. 

% Aplicabilidade dos métodos de projeção
Those results make the proposed method promissing in real life applications. One could picture this thesis technique as a tool for \gls{AI} engineers adding up diverse projection methods as a trade-off of memory and computational cost for better accuracy performance. Even though the \gls{CV} models that would be attached to the proposed method have lower-enough latence to produce a responsive application, they could not be used directly in wearables devices such as smartwaches, where there is a memory constraint. It would be necessary to process the signal in a remote device with more primary storage capacity, which would be possible in a remote healthcare environment. Therefore, this thesis method would be suitable for remote healthcare applications.	 

% Trabalhos futuros
However, there are several improvement points that future works could seek their corresponding solutions. For instance, we used only a single dataset that, per se, is very limited in size and recording length. A new work could conduct experiments in larger datasets and cross-dataset validation would produce more trustable results. Another improvement point is to test the proposed method against the SQA literature especiaized methods. That would measure the real relevance of this thesis method. Another ideia would be to combine the proposed technique with the multiscale technique of Liu et al. \cite{imaging-6}, for a possible further performance increase. Thus, there is much left for exploring about the proposed projection mixture method.
