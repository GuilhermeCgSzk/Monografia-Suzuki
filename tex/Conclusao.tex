%This work presented a study on \gls{SQA} for \gls{ppg} signals. The investigated approach consists of projecting \gls{1D} signals onto \gls{2D} images using \acrshort{rp}, \acrshort{gaf}, and \acrshort{mtf} methods. Further to these methods, we also proposed the use of a mixture of them. Results indicate that \acrshort{rp} and \gls{PM} projection methods achieved better performance than \acrshort{gaf} and \acrshort{mtf} methods, while \acrshort{rp} and \gls{PM} had similar results. While the set of machine learning models was large, the number of data available in the \gls{BUTPPG} dataset was small and unbalanced, not giving very definitive results. Therefore, the experiment needs to be replicated on a large dataset, either by using data augmentation techniques to balance and enlarge the BUT PPG dataset or by utilizing another dataset with more samples. 

% Conclusão derivada dos resultados
This thesis lead to a series of relevant conclusions. The experiments of the Chapter \ref{Experimentos} gives insights about the importance of the novel projection method proposed in the Chapter \ref{Metodologia}. One important result is that the \gls{PMix} method was more prevalent in the best results than the other projection methods, since, among the projection-based ensembles in the best results Table \ref{tab:Averages_of_best models}, all occurrences except one used the \gls{PMix} method. That leads to the conclusion that we can use the proposed method to improve the performance of a set of available isolated projection methods. However, it should be considered that that method increase the primary memory requirements to the summation of the individual projection sizes. The same experiments also allow us to notice the competence of the projection methods in general. An important experiment result on that regard is that, in the best models Table \ref{tab:Averages_of_best models}, the \gls{PMix} method with the Wide ResNet surpassed the baseline time series classification models for the Cohen Kappa score, indicating that a projection-based approach was capable of being very accurate for both binary \gls{SQI} classes. That highlights the viability of projection-based approach in par with other time series classifiers, but the conventional time series classifiers did achieved a better F1 and Precision scores, hinting that they worked better for the positive \gls{SQI} class. An additional disadvantage is that the combination of the projection-based with the \gls{CV} models is computationally more expensive and occupies more memory than 1D classifiers. Therefore the projection-based approach and the projection-based approach demonstrated to be useful in the task of the \gls{SQA}, our original objective described in the Section \ref{sec:problem}.

% Aplicabilidade dos métodos de projeção
Despite the disadvantages, the results uncover the proposed method as promising tool in real life applications that the Chapter \ref{Introducao} presented.
% - Ferramenta de engenharia de IA
One could picture this thesis technique as a tool for \gls{AI} engineers adding up diverse projection methods as a trade-off of memory and computational cost for better accuracy performance, since that technique would, respectively, add the image size and the 1D-to-2D conversion cost of each projection. 
% - Aplicação em dispositivos vestíveis
For that reason, even though the \gls{CV} models that would be attached to the proposed method have lower-enough latency to produce a responsive application, they could not be used directly in wearables devices such as smartwatches, where there is a memory constraint.
% - Aplicação em monitoramento médico remoto
It would be necessary to process the signal in a remote device with more primary storage capacity, which would be possible in servers of a remote healthcare environment. 
% - Conclusão
Therefore, this thesis method would still be suitable for remote healthcare applications.	 


% Posição do trabalho na literatura
When we consider those experimental results and the decisions that produced them, it is possible to understand the place of this work in the literature, which the Chapter \ref{Revisao_bibliografica} reviewed. 
% - Inovativo
Regarding the proposed projection method, there is no other known work that suggested that approach, which makes this work innovative. However, our experiments did not put the method against the existing \gls{SQA} works, which does give the \gls{PMix} an unsure place in the literature. 
% - Útil
Adding to its originality, our experiments tested an unusually high variety of 2D and 1D models, not common in the literature, which positions this work as useful reference for knowing models that synergizes well with the \gls{SQA} task, even though our experiments tested only one small dataset. 
% - Reprodutível
Furthermore, unlike most works in the literature, our experiments are reproducible, since not only our implementation used open-source libraries for the \gls{ML} models and projection methods, but also our experiments worked with a publicly available dataset with established labeling. Toting to the reproducibility, our implementation is available at \url{https://gitlab.com/lisa-unb/projection-based-biological-signal-processing}, although it was not refined and properly documented for external user usage, but could be read through.
% - Conclusão
Therefore, this thesis is innovative, useful and reproducible, as the Section \ref{sec:my_work} exposed.      

% Trabalhos futuros
However, there are several improvement points, some already presented in the Section \ref{sec:Limitations}, for which future works could seek their corresponding solutions. 
% - Mais datasets
For instance, we used only a single dataset that is very limited in size, data quality variety, and recording length. A new work could conduct experiments in larger and distinct datasets and cross-dataset validation would produce more trustable results. 
% - Explorar mais opções de modelos e parâmetros
Conversely, our experiments did not explore many 1D and 2D models with open-source implementations and did not varied the parameters of the projections and of the \gls{ML} models. Doing that would uncover new relationships among the models and its parameters. 
% - Comparar com métodos SQA estabelecidos
Increasing the model options, another improvement point is to test the proposed method against the SQA literature specialized methods. That would measure the real relevance of the method of this thesis. 
% - Combinar com técnicas de SQA existentes
Another idea related to the \gls{SQA} literature would be to combine the proposed method with other existing \gls{SQA} techniques, such as the signal multiscaling technique of Liu et al. \cite{imaging-6}. That would possibly further increase performance of the \gls{PMix}. 
% - Refinar implementação
Finally, our implementation could be further refined not only by better choosing pre-processing techniques and \gls{ML} training strategies, but also by refining the experimental setup and environment. 
% - Conclusão
Thus, there is much to improve in future works when considering the experimental setup and the proposed method.
