This work, \change{ named ``Projection-Based Photoplethysmography Signal Quality Assessment''}, presented a study on \gls{SQA} for \gls{PPG} signals, \change{mainly focused on the 1D-to-2D-projection approach}. The investigated \change{projection-based} approach involved \change{transforming} {1D} signals onto {2D} images using the \acrshort{RP}, \acrshort{GAF}, and \acrshort{MTF} methods. In addition to these methods, we proposed a mixed approach combining them. The results indicate that the \acrshort{RP} and \gls{PMix} projection methods outperformed the \acrshort{GAF} and \acrshort{MTF} methods, with \acrshort{RP} and \gls{PMix} yielding similar outcomes. Although the set of machine learning models was extensive, the \gls{BUTPPG} dataset was small and unbalanced, limiting the conclusiveness of the results. Consequently, the experiment should be replicated on a larger dataset, either by using data augmentation techniques to balance and expand the BUTPPG dataset or by utilizing a different dataset with more samples. \change{Nonetheless, the method of this work was published on the \textit{Anais do XXIV Simpósio Brasileiro de Computação Aplicada à Saúde} through the peer-reviewed work ``\textit{On the Performance of Composite 1D-to-2D Projections for Signal Quality Assessment}''~\cite{sbcas}, in which we tested the proposed method and the \gls{RP}, \gls{MTF}, and \gls{GAF} projection methods in the same \gls{BUTPPG} dataset with a smaller set of \gls{CV} models.}

% Conclusão derivada dos resultados
The experiments in Chapter \ref{Experimentos} provide insights into the importance of the novel projection method proposed in Chapter \ref{Projections}. One key finding is that the \gls{PMix} method appeared more frequently among the best-performing results compared to the other projection methods. Specifically, in the projection-based ensembles listed in Table~\ref{tab:Averages_of_best models}, all but one used the \gls{PMix} method. This suggests that the proposed method can enhance the performance of a set of isolated projection methods. However, it should be noted that this method increases memory requirements due to the cumulative size of the individual projections. The same experiments also demonstrate the overall effectiveness of projection methods. Notably, in the best-performing models listed in Table \ref{tab:Averages_of_best models}, the \gls{PMix} method combined with Wide ResNet outperformed the baseline time series classification models in terms of the Cohen Kappa score. This indicates that a projection-based approach can be highly accurate for both binary \gls{SQI} classes. While this highlights the viability of the projection-based approach alongside other time series classifiers, the conventional time series classifiers achieved higher F1 and Precision scores, suggesting they performed better for the positive \gls{SQI} class. An additional drawback is that combining projection-based methods with \gls{CV} models is computationally more expensive and requires more memory than using 1D classifiers. Nevertheless, the projection-based approach proved to be effective for \gls{SQA}, fulfilling our original objective described in Section~\ref{sec:problem}.


% Aplicabilidade dos métodos de projeção
The results reveal that the proposed method is a promising tool for real-life applications, as presented in Chapter~\ref{Introducao}.
% - Ferramenta de engenharia de IA
One could envision this thesis technique as a tool for \gls{AI} engineers, offering a trade-off between memory and computational cost in exchange for improved accuracy. This is achieved by combining various projection methods, which respectively increase image size and incur the 1D-to-2D conversion cost of each projection.
% - Aplicação em dispositivos vestíveis
For that reason, even though the \gls{CV} models incorporated into the proposed method have sufficiently low latency to support a responsive application, they may not be advisable for wearable devices, such as smartwatches, due to memory constraints.
% - Aplicação em monitoramento médico remoto
It is necessary to process the signal on a remote device with greater memory capacity, such as a server in a remote healthcare environment.
% - Conclusão
Therefore, this method is suitable for remote healthcare applications.


% Posição do trabalho na literatura
When considering the experimental results and the decisions that produced them, it is possible to understand the place of this work in the literature reviewed in Chapter~\ref{Revisao_bibliografica}.
% - Inovativo
Regarding the proposed projection method, there is no known work suggesting this approach, which makes our work innovative. However, since our experiments did not directly compare the method with existing \gls{SQA} approaches, the position of the \gls{PMix} method in the literature remains uncertain.
% - Útil
In addition to its originality, our experiments tested an unusually wide variety of 2D and 1D models, which is not common in the literature. This positions our work as a valuable reference for identifying models that synergize well with the \gls{SQA} task, despite the limitation of testing on only a small dataset.
% - Reprodutível
Furthermore, in contrast to most works in the literature, our experiments can be reproduced. Our implementation uses open-source libraries for both the \gls{ML} models and projection methods and works with a publicly available dataset with established labeling. Moreover, our software implementation is also publicly available\footnote{\url{https://gitlab.com/lisa-unb/projection-based-biological-signal-processing}}. Although it is not yet fully refined or thoroughly documented for external use, it is accessible for review.
% - Conclusão
Therefore, this work is innovative, useful and reproducible, as the Section~\ref{sec:my_work} exposed.

% Trabalhos futuros
There are several improvement points, some already presented in the Section \ref{sec:Limitations}, for which future works could seek their corresponding solutions. 
% - Mais datasets
For instance, we used only a single dataset, which is limited in size, data quality, variety, and recording length. Future work could involve experiments with larger and more diverse datasets, and cross-dataset validation would yield more reliable results.
% - Explorar mais opções de modelos e parâmetros
Conversely, the experiments did not explore many 1D and 2D models with open-source implementations, nor did they vary the parameters of the projections and \gls{ML} models. Exploring these aspects could reveal new relationships among the models and their parameters.
% - Comparar com métodos SQA estabelecidos
Increasing the model options, another improvement point is to test the proposed method against specialized methods from the SQA literature. This would assess the real relevance of the proposed method.
% - Combinar com técnicas de SQA existentes
Another idea related to the \gls{SQA} literature would be to combine the proposed method with other existing \gls{SQA} techniques, such as the signal multiscaling technique of Liu et al. \cite{imaging-6}. That would possibly further increase performance of the \gls{PMix}. 
% - Refinar implementação
Finally, the implementation could be further refined not only by selecting more effective pre-processing techniques and \gls{ML} training strategies but also by improving the experimental setup and environment.
% - Conclusão
Hence, there is significant potential for improvement in future work regarding both the experimental setup and the proposed method.
