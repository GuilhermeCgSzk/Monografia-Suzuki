%This work presented a study on \gls{SQA} for \gls{ppg} signals. The investigated approach consists of projecting \gls{1D} signals onto \gls{2D} images using \acrshort{rp}, \acrshort{gaf}, and \acrshort{mtf} methods. Further to these methods, we also proposed the use of a mixture of them. Results indicate that \acrshort{rp} and \gls{PM} projection methods achieved better performance than \acrshort{gaf} and \acrshort{mtf} methods, while \acrshort{rp} and \gls{PM} had similar results. While the set of machine learning models was large, the number of data available in the \gls{BUTPPG} dataset was small and unbalanced, not giving very definitive results. Therefore, the experiment needs to be replicated on a large dataset, either by using data augmentation techniques to balance and enlarge the BUT PPG dataset or by utilizing another dataset with more samples. 

This thesis lead to a series of conclusions. The main conclusion of this work is that the proposed projection method produced equal or better results than the other existing projections. For that reason, we can use the projection mixture as a tool for improving performance of projection-based frameworks at the cost of a small increase in execution time and memory usage. Another relevant conclusion is that the projection-based approach for SQA promoted more accurate results than time series classification baseline models, at the cost of computational expense. With that in mind, organizations with less resources can opt to employ the best baseline models of this work, while ones with more resources can employ the 2D approach. Therefore, this thesis produced useful results. 

However, there are several improvement points that future works could seek their corresponding solutions. For instance, we used only a single dataset that, per se, is very limited in size and recording length. A new work could conduct experiments in larger datasets and cross-dataset validation would produce more trustable results. Another improvement point is to test the proposed method against the SQA literature especiaized methods. That would measure the real relevance of this thesis method. Another ideia would be to combine the proposed technique with the multiscale technique of ? Et al, for a possible further performance increase. Thus, there is much left for exploring about the proposed projection mixture method.
