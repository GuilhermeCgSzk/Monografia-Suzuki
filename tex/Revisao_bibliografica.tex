
% Supõe-se que, na introdução, tenha sido posto a definição do problema de SQA como a necessidade de garantir um sinal de boa qualidade, seja para humanos não cometerem enganos, seja para modelos preditivos serem capazes de estimar com acurácia.

In this chapter I present the state of the literature of the \acrlong{SQA} of cardiological signals problem, focused on my work scope.

\section{The Feature Extraction Method}
\label{sec:feature}

To give a proper \acrshort{SQI} for the \acrshort{SQA}, we need to extract proper features from the input signal. For this purpose, researchers proposed several features to be extracted. For instance, \citeauthor{review-1} poses the measurement of a \acrshort{SQI} through the application of the Dynamic Time Warping techinique \cite{review-1}. It finds the optimal path cost on a distance matrix which encodes the differences between each pair of points of the input and a template, reference of a good signal. \citeauthor{review-1} added this techinique \acrshort{SQI} to others and feeded them to a \acrshort{MLP} and to a self-made function, predicting a unique \acrshort{SQI}. Experiments on private annotations on the MIMIC II dataset resulted on the \acrshort{MLP} achieving the highest accuracy, 95.2\% \cite{review-1}. However, the result analysis does not isolate the proposed \acrshort{SQI}, not making clear how much it contributed to the final result if compared to the other \acrshort{SQI}s.      

Later, \citeauthor{review-2} proposed two features for the estimation of a classification \acrshort{SQI} of multi-channeled \acrshort{ECG}s. One feature consists of verifying if two energy-like indices, measured in deciBels, are within an admissible range. The first converts, for an instant of time, the signal amplitude to a logarithm scale. The second originates from a similar conversion, but by measuring the signal concavity. The other feature result from randomly chosing a targed lead, feeding a \acrshort{FFNN} with array of derivatives of all leads to reconstruct the targed lead and finally comparing the original targed lead to its artificial version with correlation analysis. The researchers submited an entry on the CinC Challenge 2011 and achieved a accuracy of 93.60\%, overperforming many other approaches of its time.  

In 2017, \citeauthor{review-3} introduces a feature based on the extraction of the Heart Rate Variability from \acrshort{ECG} signals. The method decomposes this new signal into wavelets with different frequency ranges and calculates their entropy, forming a feature vector. Those features feed an \acrshort{SVM} 	 

\section{The Arythmia Problem}
\label{sec:arythmia}

\section{The Deep Learning Approach}
\label{sec:deep_learning}

\section{This Time Series Imaging Techinique}
\label{sec:imaging}

\section{The Matrix Embedding Techinique}
\label{sec:matrix}
