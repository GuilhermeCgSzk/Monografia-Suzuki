% Introdução
Com o rápido aumento na popularidade de dispositivos vestíveis, como \textit{smartwatches}, aplicativos de monitoramento de saúde estão ganhando destaque. Esses aplicativos costumam utilizar esse tipo de dispositivo para captar sinais úteis no diagnóstico das condições de saúde de um indivíduo, como o fotopletismograma. O método de obtenção desse tipo de sinal, a fotopletismografia, é compacto, não-invasivo e econômico, sendo por isso bastante vantajoso. Apesar de seus benefícios, a fotopletismografia tem a grande limitação de ser particularmente suscetível a artefatos de movimento e interferências ambientais. Esses problemas podem deteriorar a qualidade do sinal, o que prejudica significativamente a eficácia dos aplicativos que o consomem. Portanto, avaliar a qualidade dos sinais é essencial nas aplicações de monitoramento de saúde.
% Resumo do trabalho
Para esse fim, algoritmos de aprendizado de máquina podem ser aplicados. Este trabalho apresenta um método inovador para avaliar a qualidade dos sinais de fotopletismografia, realizado através da fusão de projeções de sinais e técnicas de visão computacional. Para ser mais preciso, o sinal unidimensional é projetado em um conjunto de representações bidimensionais. Isso pode ser feito usando técnicas de imagem de séries temporais, como \textit{Gramian Angular Field}, \textit{Markov Transition Field} e \textit{Recurrence Plots}, além de agregar seus resultados, o que chamamos de `\textit{Projection Mix}'. Com essas projeções, várias modelos de visão computacional foram treinadas e testadas na base de dados \textit{\acrshort{BUTPPG}}, com hiperparâmetros selecionados através de busca heurística. Os resultados indicaram que o \textit{Recurrence Plot} e o \textit{Projection Mix} geralmente superaram as outras projeções usadas no estudo. Além disso, os métodos baseados em projeção alcançaram resultados comparáveis a classificadores 1D de séries temporais. Por exemplo, a combinação de \textit{Wide ResNet} com \textit{Projection Mix} alcançou uma pontuação média de \textit{Cohen Kappa} de 95,5\% (remapeada de $[-1,1]$ para $[0,1]$) com um desvio padrão de 0,101. 
%A implementação do método e dos experimentos descritos nesta tese pode ser encontrada em \url{https://gitlab.com/lisa-unb/projection-based-biological-signal-processing}.
