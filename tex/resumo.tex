A avaliação da qualidade de sinais é essencial no monitoramento de saúde contínuo, visto que sinais de boa qualidade são necessários para inferir com segurança as condições médicas do paciente. Para essa finalidade, podemos aplicar modelos de visão computacional. No entanto, para esses modelos, é necessário transformar o sinal 1D para uma representação 2D. Isso pode ser feito usando técnicas de projeção de sinais, como o \acrlong{GAF}, \acrlong{MTF} e o \acrlong{RP}, além de agregá-los em um único composto, o qual chamamos de \acrlong{PMix}. Após processar o conjunto de dados, \acrlong{BUTPPG}, para produzir essas imagens, treinamos e testamos modelos de visão computacional, além de selecionar hiperparâmetros com busca heurística. Os resultados mostram que as projeções \acrlong{PMix} e \acrlong{RP} superaram as projeções \acrlong{GAF} e \acrlong{MTF} para a maioria dos modelos de visão computacional. Além disso, as técnicas de projeção alcançaram resultados que competem com classificadores 1D de séries temporais. Por exemplo, a combinação da Wide ResNet com \acrlong{PMix} alcançou $95.5\%$ (remapeado de $[-1,1]$ para $[0,1]$) para a média da pontuação de Cohen Kappa dos $K$ folds produzidos, com um desvio padrão de $0.101$. A implementação do método e dos experimentos descritos nesta tese podem ser encontrados no link \url{https://gitlab.com/lisa-unb/projection-based-biological-signal-processing}.
