Com o rápido aumento na popularidade de dispositivos vestíveis, como smartwatches, aplicativos de monitoramento de saúde estão ganhando destaque devido à sua capacidade de monitorar uma variedade de métricas de saúde, como padrões de sono, frequência cardíaca e atividade física. Esses aplicativos costumam utilizar sensores de fotopletismografia para avaliar vários aspectos da saúde e bem-estar de um indivíduo. A fotopletismografia é um método óptico não invasivo e econômico que detecta variações no volume sanguíneo dentro da rede microvascular dos tecidos, fornecendo medições contínuas das mudanças fisiológicas ao longo do tempo. Examinar sinais de fotopletismografia permite a extração de informações valiosas sobre a saúde cardiovascular e várias métricas fisiológicas, incluindo variabilidade da frequência cardíaca, saturação periférica de oxigênio e padrões de sono. No entanto, apesar de seus benefícios, a fotopletismografia tem uma grande limitação: é particularmente suscetível a artefatos de movimento e interferências ambientais. Esses problemas podem prejudicar significativamente a eficácia dos aplicativos baseados em fotopletismografia, especialmente ao capturar sinais de fotopletismografia usando dispositivos vestíveis. Portanto, para obter medições precisas, é crucial ter sinais apropriados que sejam amostrados com alta confiabilidade.

Nesse contexto, avaliar a qualidade dos sinais é essencial para permitir a aplicação de monitoramento de saúde, já que uma alta qualidade do sinal é crucial para avaliar de forma confiável a condição médica do paciente. Para alcançar isso, algoritmos de aprendizado de máquina podem ser aplicados. Este trabalho apresenta um método inovador para avaliar a qualidade dos sinais de fotopletismografia, realizado através da fusão de projeções de sinais e técnicas de visão computacional. Para ser mais preciso, o sinal de fotopletismografia unidimensional é projetado em um conjunto de representações bidimensionais. Isso pode ser feito usando técnicas de imagem de séries temporais, como Gramian Angular Field, Markov Transition Field e Recurrence Plots, além de agregar seus resultados, o que chamamos de `Projection Mix'. Após o pré-processamento do conjunto de dados \gls{BUTPPG} em essas imagens, várias redes neurais profundas são treinadas e testadas, com hiperparâmetros selecionados através de busca heurística. Os resultados indicam que o Recurrence Plot e o Projection Mix geralmente superaram o Gramian Angular Field e o Markov Transition Field na maioria dos modelos de visão computacional. Além disso, os métodos baseados em projeção alcançaram resultados comparáveis aos classificadores 1D de séries temporais. Por exemplo, a combinação de Wide ResNet com Projection Mix alcançou uma pontuação média de Cohen Kappa de 95,5% (remapeada de $[-1,1]$ para $[0,1]$) com um desvio padrão de 0,101. A implementação do método e dos experimentos descritos nesta tese pode ser encontrada em \url{https://gitlab.com/lisa-unb/projection-based-biological-signal-processing}.