
\newcommand{\projectionsWidth}{0.13\textwidth}

\begin{figure}[h]
	\centering
	\bgroup
	\setlength{\tabcolsep}{1mm}	
	\def\arraystretch{2}
	\begin{tabular}{cccc}
		\multicolumn{4}{c}{\includegraphics[width=0.6\textwidth, trim={2cm 0cm 2cm 0cm}]{img/projections/signal.png}} \\
		\fbox{\includegraphics[width=\projectionsWidth]{img/projections/GramianAngularFieldDifference.png}}	
		 & \fbox{\includegraphics[width=\projectionsWidth]{img/projections/GramianAngularFieldSummation.png}}
		 & \fbox{\includegraphics[width=\projectionsWidth]{img/projections/MarkovTransitionField.png}}		
		 & \fbox{\includegraphics[width=\projectionsWidth]{img/projections/RecurrencePlot.png}}	\\
		\fbox{\includegraphics[width=\projectionsWidth]{img/projections/PoincatePlotLogarithmGrid.png}}	
		& \fbox{\includegraphics[width=\projectionsWidth]{img/projections/MultiscaleMarkovTransitionField.png}}
		& \fbox{\includegraphics[width=\projectionsWidth]{img/projections/ShortTimeFFT.png}}
		& \\	
	\end{tabular}
	\egroup
	\caption{A signal and its various projection obtained by several methods. In the first line, from the left to the right, the methods are: Gramian Angular Diference Field, Gramian Angular Summation Field, Markov Transition Field and Recurrence Plot. The methods of the second line are, from the left to the right: Poincaré Plot Density Map, Multiscale Markov Transition Field and Short Time Fourier Transform Spectogram}
	\label{fig:literature_projections}
\end{figure}
