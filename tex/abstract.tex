Signal quality assessment is essential to health monitoring applications since good signal quality is needed to inform reliably the medical conditions of the patient. In order to do so, machine learning algorithms such as convolutional neural networks may be applied. However, the signal needs to be transformed into a 2D representation, which can be done by the use of time series imaging techniques, such as Gramian Angular Field, Markov Transition Field, and Recurrence Plot, and also by aggregating their results, which we called Projection Mix. After preprocessing the dataset, BUT PPG, into those images, various convolutional neural networks were trained and tested using such data, while also choosing hyperparameters using heuristic searching. The results reveal that the projections Recurrence Plot and Projection Mix performed generally better than the Gramian Angular Field and Markov Transition Field, despite the existence of some exceptions. An implementation of the method described in this paper can be found at \url{https://gitlab.com/lisa-unb/projection-based-biological-signal-processing}.
