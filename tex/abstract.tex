
With the rapid rise in popularity of wearable devices like smartwatches, health monitoring applications are gaining traction. 
\textcolor{red}{Those applications commonly utilize wearable devices to record signals that are useful in the individual's health condition diagnostic, such as the photoplethysmogram. That signal extraction method, the photoplethysmography, is compact, non-invasive, and economical. Despite those benefits, the photoplethysmography is particularly susceptible to motion artifacts and environmental interferences. Those issues can greatly impair quality of the signal, which compromises the performance of the applications that consume it. Therefore  assessing the signal quality is essential for enabling health monitoring applications. }
%
To achieve this, machine learning algorithms can be applied. This work presents an innovative method for assessing the quality of photoplethysmograph signals, accomplished through a fusion of signal projections and computer vision techniques. To be more precise, the one-dimensional photoplethysmograph signal is projected to a set of bidimensional representations. This can be accomplished using time series imaging techniques, such as \glsxtrlong{GAF}, \glsxtrlong{MTF} and \glsxtrlong{RP}, and by aggregating their results, which is referred to as ``Projection Mix''. \textcolor{red}{We combined those projections with several computer vision models. Then, we trained and tested them on the \glsxtrlong{BUTPPG}}, with hyperparameters selected through heuristic searching. The results indicate that the \glsxtrlong{RP} and Projection Mix generally outperformed \glsxtrlong{GAF} and \glsxtrlong{MTF} across most compute vision models. Additionally, projection-based methods achieved results comparable to 1D time series classifiers. For instance, the combination of Wide ResNet with Projection Mix achieved a K-Fold mean Cohen Kappa score of 95.5\% (rescaled from $[-1,1]$ to $[0,1]$) with a standard deviation of 0.101. 
%An implementation of the method and experiments described in this thesis can be found at \url{https://gitlab.com/lisa-unb/projection-based-biological-signal-processing}.
