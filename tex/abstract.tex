\gls{SQA} is essential to health monitoring applications since good signal quality is needed to inform reliably the medical conditions of the patient. To do so, we may apply \gls{CV} models. However, for that kind of model, the 1D signal needs to be transformed into a 2D representation. It can be done by the use of time series imaging techniques, such as \gls{GAF}, \gls{MTF}, and \gls{RP}, and also by aggregating their results, which we called \gls{PMix}. After pre-processing the dataset, \gls{BUTPPG}, into those images, we trained and tested various \gls{CV} networks using such data, while also choosing hyperparameters using heuristic searching. The results reveal that the projections \gls{RP} and \gls{PMix} performed generally better than the \gls{GAF} and \gls{MTF} for most \gls{CV} models. Additionally, the projection-based methods achieved results that compete with 1D time series classifiers. For example, the combination of Wide ResNet with \gls{PMix} reached a K-Fold mean Cohen Kappa score of $95.5\%$ (rescaled from $[-1,1]$ to $[0,1]$) with an standard deviation of $0.101$. An implementation of the method described and of the experiments in this thesis can be found at \url{https://gitlab.com/lisa-unb/projection-based-biological-signal-processing}.
